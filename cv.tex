\documentclass{resume} % Use the custom resume.cls style

\usepackage[left=0.4in, top=0.4in, right=0.4in, bottom=0.4in]{geometry} % Document margins
\newcommand{\tab}[1]{\hspace{.2667\textwidth}\rlap{#1}}
\newcommand{\itab}[1]{\hspace{0em}\rlap{#1}}
\name{Denys Smirnov} % Your name

\address{
    +380(66)253-4992 \\
    \href{mailto:dendi239+resume@gmail.com}{dendi239@gmail.com} \\ \href{https://www.linkedin.com/in/denys-smirnov/}{linkedin} \\
    \href{https://www.github.com/dendi239}{github} \\
    \href{https://codeforces.com/profile/dendi239}{codeforces}
}  % Your phone number, email, linkedin, and (optional) website

\begin{document}

%----------------------------------------------------------------------------------------
%	OBJECTIVE
%----------------------------------------------------------------------------------------

\begin{rSection}{Objective}

Software Engineer with 3+ years of commercial experience.
Has strong olympiad mathematical and competitive programming background.
Focused on reducing routine and creating best products with available resources.

\end{rSection}
%----------------------------------------------------------------------------------------
%	EDUCATION SECTION
%----------------------------------------------------------------------------------------

\begin{rSection}{Education}
{\bf Bachelor of Mathematics}, V. N. Karazin Kharkiv National University \hfill {2015 -- 2021}

\end{rSection}

\begin{rSection}{Achievements}
\begin{itemize}
    \itemsep -3pt
    \item The Summer School Programming - Uzhgorod --- 3rd place (out of 70 teams-participants)    \hfill Aug 2020
    \item ICPC 2016, 2017, 2019 --- Semi-final participation    \hfill Oct 2016, 2017, 2019
    \item KPI-Open 2019 --- 3rd place (out of over 120 teams-participants)   \hfill Jul 2019
    \item Iran's scientific olympiad --- Bronze                 \hfill Jul 2017
    \item IMO 2015 --- \href{https://www.imo-official.org/participant_r.aspx?id=25121}{Gold Medal} (even with 1pt for P1)        \hfill Jul 2015
    \item IMO 2014 --- \href{https://www.imo-official.org/participant_r.aspx?id=25121}{Silver Medal}        \hfill Jul 2014
\end{itemize}
\end{rSection}

%----------------------------------------------------------------------------------------
% TECHINICAL STRENGTHS
%----------------------------------------------------------------------------------------
\begin{rSection}{EXPERIENCE}

\textbf{C++ Game Developer} \hfill Apr 2020 - Now \\
\href{https://playwing.com}{Playwing Ltd} \hfill \textit{Kharkiv, Ukraine}
 \begin{itemize}
    \itemsep -3pt {}
    \item Worked with Unreal Engine 4 on AAA game.
    \item Implemented stand-alone deep-linking subsystem independed from concrete UI-components code.
    \item Created responsive interface on top of Unreal Engine's widget system.
 \end{itemize}

\textbf{iOS Software engineer} \hfill Jun 2017 -- Mar 2020 \\
\href{https://cruxlab.com/}{Cruxlab Inc.} \hfill \textit{Kharkiv, Ukraine}
\begin{itemize}
    \itemsep -3pt {}
    \item Worked with various technologies: CoreData, ARKit, SceneKit, Firebase, Google Maps API, etc
    \item Mastered a lot of techniques: reactive functional programming, dependency injection, interactive transitions.
    % \item Setup continuous interactions with proper secrets handling and different scenariouses.
    \item Mastered a lot of software architectural patterns: MVC, MVVM, VIPER, RIBs.
\end{itemize}

\end{rSection}

%----------------------------------------------------------------------------------------
%	WORK EXPERIENCE SECTION
%----------------------------------------------------------------------------------------

\begin{rSection}{Projects}
\vspace{-1.25em}
\item \href{https://github.com/dendi239/yet-another-poll-bot}{\textbf{yet another poll bot.[go]}} {
    Telegram bot for polls with restrictions.
    Designed to use syntax like $!1 \& !2 \Rightarrow 3 | 4$ to check if selected options are a valid variant.
}
\item \href{https://github.com/dendi239/cp-tool}{\textbf{cp-tool.[rust]}} {
    Competitive programming tool to use with different judge systems as \href{https://ejudge.ru}{ejudge}, \href{https://codeforces.com}{codeforces}, \href{https://contest.yandex.ru}{yandex-contest}, and so on.
    Inspired by \href{https://github.com/xalanq/cf-tool}{cf-tool}, but keep reusablity for different judge systems in mind.
    % It's my first project in rust, so I tried to use as much benefits for type system as possible.
}
\item \href{https://github.com/dendi239/algorithms-data-structures}{\textbf{Algorithms and Data Structures.[cpp]}} {
    Implemented a lot of algorithms widely used in competitive programming in C++. Focused on re-usability and as much usages for as generous situations as possible with performance in mind.
}
\end{rSection}

%----------------------------------------------------------------------------------------
\begin{rSection}{Articles}

Competitive programming oriented articles. Focused on reducing amount of written code and oriented on using all available c++ features to do so.
\href{https://codeforces.com/blog/entry/79446}{General tips \& tricks for cpp source} contains various tips can reduce amount of written code: from simple typealiases to using generic lambda syntax for comparators defining.
\href{https://codeforces.com/blog/entry/79066}{Reading} describes how to define and read variables with single macro.
\href{https://codeforces.com/blog/entry/79100}{For macro} describing ways to reduce code spent on simple \texttt{for}: either a macro, or something like \href{https://en.cppreference.com/w/cpp/ranges}{ranges library} comes in c++20.

\end{rSection}


% \begin{rSection}{SKILLS}

% \begin{tabular}{ @{} >{\bfseries}l @{\hspace{6ex}} l }
% Technical Skills & A, B, C, D
% \\
% Soft Skills & A, B, C, D\\
% XYZ & A, B, C, D\\
% \end{tabular}\\
% \end{rSection}

\end{document}
